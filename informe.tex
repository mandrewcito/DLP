\documentclass[]{scrartcl}

%opening
\title{An\'alisis de las carater\'isticas de un conjunto de lenguajes de programaci\'on a partir de un caso pr\'actico}
\author{}

\begin{document}

\maketitle

\begin{abstract}

\end{abstract}

\section{Lenguaje original (Python)}
Como lenguaje original hemos elegido Python, por su sencillez ya que es pr\'acticamente pseudoc\'odigo, hemos creado un objeto aut\'omata. para guardar el aut\'omata y sus transiciones, que tambi\'en les hemos creado un objeto para ellas.
\subsection{Pseudocódigo estructuras}
Objeto Automata:\\ \
\indent estados:Lista\\ \
\indent	afabeto:Lista\\ \
\indent	estadosFinales:Lista\\ \
\indent	estadoInicial:String\\ \
\indent	Transiciones:Lista<Transiciones>\\ \
Objeto Transicion: \\ \
\indent	inicial:String\\ \
\indent	final:String\\ \
\indent	simbolo:String\\ \

funcion inicializarAutomata(fichero configuracion):\\ \
\indent	String linea = Primera linea (fichero configuracion)\\ \
\indent	String lineas[] = Dividir linea por ";"\\ \
\indent	estados=lineas[1]; alfabeto = lineas[2];\\ \
\indent	estadosFinales=Lineas[3];estadoInicial=Linea[4];\\ \
\indent	Desde i=5 hasta tamaño(linea) hacer :\\ \
\indent\indent	Añadir a Transiciones(nueva transicion linea[i]);\\ \

funcion conexo():\\ \
\indent    colaEstados=Lista vacia\\ \
\indent    colaEstados.añadir(estadoInicial)\\ \
\indent    nuevasTransiciones=Lista vacia\\ \
\indent    visitados=Lista vacia\\ \
\indent    mientras (tamaño(colaEstados)!=0):\\ \
\indent\indent      elem=colaEstados.pop()\\ \
\indent\indent      visitados.añadir(elem)\\ \
\indent\indent      para trans in Transiciones:\\ \
\indent\indent\indent        si (trans.ini==elem):\\ \
\indent\indent\indent\indent          nuevasTransiciones.añadir(trans)\\ \
\indent\indent\indent          si(trans.fin no esta en  visitados ni en colaEstados):\\ \
 \indent\indent\indent\indent           colaEstados.añadir(trans.fin)\\ \
 \indent   Transiciones=nuevasTransiciones\\ \
 \subsection{Compilador versi\'on y el sistema operativo empleado}
 Python 2.7.6, Ubuntu 64b 14.04
  \subsection{Caracter\'isticas generales del lenguaje}
  Se trata de un lenguaje de programación multiparadigma, ya que soporta orientación a objetos, pr																												ogramación imperativa y, en menor medida, programación funcional. Es un lenguaje interpretado, usa tipado dinámico y es multiplataforma.
  \subsection{An\'alisis detallado}
  F\'acil lecutra ( casi pseudocodigo), todas las listas son iterables no necesitas crear un iterador, ventaja de debug en consola de modo interactivo
  las listas ademas pueden contener varios tipos de datos no como en java, hasta pueden contener tuplas. como lenguaje original es una buena eleccion ya
   que es de muy alto nivel, la entrada salida es sencilla. tiene la desventaja de que no podemos gestionar la memoria en ningun momento .
  
  \subsection{Justificaciones respecto a las decisiones de diseño e implementaci\'on}
  Diseñaremos el automata con dos objetos, el resto de lenguaje seran de mas bajo nivel que python, tendremos que utilizar mas recursos
  \subsection{Caracter\'isticas que no se pudieron aprovechar del lenguaje}
  \section{Java}
   \subsection{Compilador versi\'on y el sistema operativo empleado}
   JAVA 8, Ubuntu 64b 14.04
    \subsection{Caracter\'isticas generales del lenguaje}
    Java es un lenguaje de programación de propósito general, concurrente, orientado a objetos y basado en clases que fue diseñado específicamente para tener tan pocas dependencias de implementación como fuera posible.
    En Java el problema fugas de memoria se evita en gran medida gracias a la recolección de basura
    \subsection{An\'alisis detallado}
    muy parecido al lenguaje original, con la diferencia de las listas, que necesitan un iterador para recorrerlas y no se pueden iterar
    directamente como en python, además la entrada salida se tiene que manejar a mayores con excepciones y en general es un poco mas laborioso que
    en python, en cuanto a la orientacion a objetos es igual. solo se han cambiado las listas de python por los arrayList de java
    \subsection{Justificaciones respecto a las decisiones de diseño e implementaci\'on}
    diseñaremos el automata con dos objetos, con la modificacion de las listas que ahora son array list.
    \subsection{Caracter\'isticas que no se pudieron aprovechar del lenguaje}
      \section{C}
       \subsection{Compilador versi\'on y el sistema operativo empleado}04
        \subsection{Caracter\'isticas generales del lenguaje}
        \subsection{An\'alisis detallado}
        \subsection{Justificaciones respecto a las decisiones de diseño e implementaci\'on}
        \subsection{Caracter\'isticas que no se pudieron aprovechar del lenguaje}
          \section{Ensamblador}
           \subsection{Compilador versi\'on y el sistema operativo empleado}04
            \subsection{Caracter\'isticas generales del lenguaje}
            \subsection{An\'alisis detallado}
            \subsection{Justificaciones respecto a las decisiones de diseño e implementaci\'on}
            \subsection{Caracter\'isticas que no se pudieron aprovechar del lenguaje}
\end{document}
