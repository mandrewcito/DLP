% Usar el tipo de documento: Artículo científico.
\documentclass[12pt,a4paper]{article}

% Cargar mensajes en español.
\usepackage[spanish]{babel}

% Usar codificación utf-8 para acentos y otros.
\usepackage[utf8]{inputenc}

%Dimensiones de los márgenes.
\usepackage[margin=1.5cm]{geometry}

% Insertar porciones de código
\usepackage{listings}

% Comenzar párrafos con separación no indentación.
\usepackage{parskip}

% Usar gráficos
\usepackage{graphicx}
\usepackage{caption}
\usepackage{subcaption}
%
% Usar contenedores flotantes para figuras.
\usepackage{float}

% Carpeta de las imágenes.
\graphicspath{{img/}}

% Configuración para porciones de código.
\lstset{
%	language=bash,
	basicstyle=\ttfamily\small,
%	numberstyle=\footnotesize,
%	numbers=left,
%	backgroundcolor=\color{gray!10},
%	frame=single,
	tabsize=4,
%	rulecolor=\color{black!30},
%	title=\lstname,
%	escapeinside={\%*}{*)},
	breaklines=true,
	breakatwhitespace=true,
%	framextopmargin=2pt,
%	framexbottommargin=2pt,
	extendedchars=false,
	inputencoding=utf8
}
%%%%%%%%%%%%%%%%%%%%%%%%%%%%%%%%%%%%%%%%%%%%%%%%%%%%%%%%%%%%%%%%%%%%%%%%%%%%%%%

% Propiedades
\title{Diseño de un lenguaje de Programación}

\author{Andrés Baamonde Lozano (andres.baamonde@udc.es)\\
	Rodrigo Arias Mallo (rodrigo.arias@udc.es)}

\begin{document}

\maketitle

%%%%%%%%%%%%%%%%%%%%%%%%%%%%%%%%%%%%%%%%%%%%%%%%%%%%%%%%%%%%%%%%%%%%%%%%%%%%%%%
\clearpage 

\tableofcontents

\clearpage 

\section{Introducción}
Este lenguaje, que está enfocado a las operaciones matemáticas pretende utilizar al máximo la capacidad de la GPU que tenga el ordenador que lo ejecute para así liberar de cálculos a la CPU y así hacer  una ejecución más eficiente. A grandes rasgos, vamos a hacer un lenguaje como C pero añadiendole funcionalidades como multiplicación de matrices.
\section{Paradigmas}
Programación imperativa utilizando funciones.
\section{Gestión de memoria}
Manual, se especifican tamaños o se inicializan las variables con un valor determinado.
\section{Sistema de tipos}
Tipado estático, deben especificarse todos los tipos de las variables. En las matrices se deben especificar sus dimensiones y su tipo, que sólo puede ser uno no se permiten matrices heterogeneas. 
\section{Tipos de dato}
\subsection{Básicos}
\begin{itemize}
\item Int.
\item Double.
\item Float.
\item Char.
\item Punteros a tipos básicos(como en C).
\end{itemize}
\subsection{Complejos}
\begin{itemize}
\item Vectores.
\item Matrices.
\end{itemize}
\section{Operadores}
\subsection{Tabla de operadores}
Mismos que en C.
\subsection{Sobrecarga operadores}
Se añaden (sobrecargando los operadores) las operaciones con vectores y matrices.
\section{Visibilidad}
\subsection{Visibilidad Local}
A nivel de función.
\subsection{Visibilidad Global}
A nivel de programa.
\section{Tratamiento de errores}
\subsection{Matemáticos}
Se lanza una excepción que interrumpe el programa automáticamente (Lo hace el compilador).
Errores :
\begin{itemize}
\item algo
\end{itemize}
\subsection{Overflow}
Se lanza una excepción que interrumpe el programa automáticamente (Lo hace el compilador).
Errores :
\begin{itemize}
\item algo
\end{itemize}
\subsection{Segmentación}
Se lanza una excepción que interrumpe el programa automáticamente (Lo hace el compilador).
Errores :
\begin{itemize}
\item algo
\end{itemize}
\section{Programas o trozos de código}
\section{Ventajas y desventajas}
\subsection{Optimizaciones en cálculos}
Añadimos el paralelismo de la GPU para agilizar los cálculos de matrices y vectores. 
\section{Necesidades especiales}
\subsection{IDE}
\subsection{Puntos de interrupción}
Se puede compilar de forma secuencial en la CPU para así poder depurar sin problemas y después poder optimizar el cálculo haciéndolo de forma paralela.
\subsection{Gráfica}
Necesidad de una targeta gráfica que soporte OpenCL para poder utilizar la ejecución en paralelo.
\end{document}
