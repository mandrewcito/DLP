% Usar el tipo de documento: Artículo científico.
\documentclass{beamer}

% Cargar mensajes en español.
\usepackage[spanish]{babel}

% Usar codificación utf-8 para acentos y otros.
\usepackage[utf8]{inputenc}

%Dimensiones de los márgenes.
%\usepackage[margin=1.5cm]{geometry}

% Insertar porciones de código
\usepackage{listings}

% Comenzar párrafos con separación no indentación.
\usepackage{parskip}
%enlaces
\usepackage{hyperref}
% Usar gráficos
\usepackage{graphicx}
\usepackage{caption}
\usepackage{subcaption}
%
% Usar contenedores flotantes para figuras.
\usepackage{float}

% Carpeta de las imágenes.
\graphicspath{{img/}}

% Configuración para porciones de código.
\lstset{
%	language=bash,
	basicstyle=\ttfamily\small,
%	numberstyle=\footnotesize,
%	numbers=left,
%	backgroundcolor=\color{gray!10},
%	frame=single,
	tabsize=4,
%	rulecolor=\color{black!30},
%	title=\lstname,
%	escapeinside={\%*}{*)},
	breaklines=true,
	breakatwhitespace=true,
%	framextopmargin=2pt,
%	framexbottommargin=2pt,
	extendedchars=false,
	inputencoding=utf8
}
%%%%%%%%%%%%%%%%%%%%%%%%%%%%%%%%%%%%%%%%%%%%%%%%%%%%%%%%%%%%%%%%%%%%%%%%%%%%%%%

% Propiedades
\title{Lenguaje de programación para cálculo paralelo.}

\author{Andrés Baamonde Lozano (andres.baamonde@udc.es)\\
	Rodrigo Arias Mallo (rodrigo.arias@udc.es)}

\begin{document}

%\maketitle

%%%%%%%%%%%%%%%%%%%%%%%%%%%%%%%%%%%%%%%%%%%%%%%%%%%%%%%%%%%%%%%%%%%%%%%%%%%%%%%
%\clearpage 

%\tableofcontents

%\clearpage 
\frame{\titlepage}
\begin{frame}
\frametitle{Introducción}
\begin{block}{Propósito}
\begin{itemize}
\item Lenguaje de programación orientado al cálculo numérico.
\item Objetivo: explotar al máximo la GPU y la CPU.
\end{itemize}
\end{block}
\pause
\begin{block}{Lenguajes relacionados}
\begin{itemize}
\item C, que implica una programación cuidadosa.
\item OpenCL, cuyo objetivo es ser un estándar abierto en el ámbito computación paralela.
\item Matlab, para las funciones sobre imágenes.
\end{itemize}
\end{block}
\end{frame}

\begin{frame}
\frametitle{Paradigma}
\begin{itemize}
\item Programación imperativa.
\item Necesaria sencillez para operar con la GPU.
\item Descartados otros paradigmas, no es necesario un alto nivel de abstracción.
\end{itemize}
\end{frame}

\begin{frame}
\frametitle{Gestión de memoria}
\begin{itemize}
\item A cargo del programador(como C).
\item Gestión de memoria dinámica retiene la memoria
más tiempo del necesario.
\end{itemize}
\end{frame}
\begin{frame}
\frametitle{Tipado}
\begin{itemize}
\item Tipado estático.
\item Se establece para las variables dimensión y tipo.
\item Incrementa el tiempo de desarrollo.
\item En tiempo de ejecución requiere menos comprobaciones.
\end{itemize}
\end{frame}
\begin{frame}
\frametitle{Tipos y modificadores}
\begin{block}{Tipos}
\begin{itemize}
\item Básico(C).
\item Complejos(Vector, Matriz,imaginario).
\end{itemize}
\end{block}
\pause
\begin{block}{Modificadores}
Además de los modificadores a nivel de función (Local) y a nivel de programa(Global).Existirán unos modificadores de tipos para cargar las variables en GPU o CPU.
\end{block}
\end{frame}
\begin{frame}
\frametitle{Operadores}
\begin{itemize}
\item Tradicionales de C.
\item Sobrecarga de operadores para los tipos complejos.
\end{itemize}
\end{frame}
\begin{frame}
\frametitle{Errores y características}
\begin{block}{Errores}
\begin{itemize}
\item Matemáticos.
\item Desbordamiento.
\item Segmentación.
\end{itemize}
\end{block}
\pause
\begin{block}{Características}
\begin{itemize}
\item Inserción de código OpenCL.
\item Evaluación en cortocircuíto.
\end{itemize}
\end{block}
\end{frame}
\end{document}
