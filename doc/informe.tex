% Usar el tipo de documento: Artículo científico.
\documentclass[12pt,a4paper]{article}

% Cargar mensajes en español.
\usepackage[spanish]{babel}

% Usar codificación utf-8 para acentos y otros.
\usepackage[utf8]{inputenc}

% Aprovechar más los márgenes de la página
\usepackage{fullpage}

% Insertar porciones de código
\usepackage{listings}

% Comenzar párrafos con separación no indentación
\usepackage{parskip}

% Configuración para porciones de código
\lstset{
%	language=bash,
	basicstyle=\ttfamily\small,
%	numberstyle=\footnotesize,
%	numbers=left,
%	backgroundcolor=\color{gray!10},
%	frame=single,
	tabsize=4,
%	rulecolor=\color{black!30},
%	title=\lstname,
%	escapeinside={\%*}{*)},
	breaklines=true,
	breakatwhitespace=true,
%	framextopmargin=2pt,
%	framexbottommargin=2pt,
	extendedchars=false,
	inputencoding=utf8
}

%%%%%%%%%%%%%%%%%%%%%%%%%%%%%%%%%%%%%%%%%%%%%%%%%%%%%%%%%%%%%%%%%%%%%%%%%%%%%%%

% Propiedades
\title{Análisis de las caraterísticas de un conjunto de lenguajes de
programación a partir de un caso práctico}

\author{Andrés Baamonde Lozano (andres.baamonde@udc.es)\\
	Rodrigo Arias Mallo (rodrigo.arias@udc.es)}

\begin{document}

\maketitle

%%%%%%%%%%%%%%%%%%%%%%%%%%%%%%%%%%%%%%%%%%%%%%%%%%%%%%%%%%%%%%%%%%%%%%%%%%%%%%%

\section{Descripción}
En el presente documento se describen varias soluciones a un mismo problema,
realizadas en diferentes lenguajes de programación. El problema consiste en
eliminar los estados finales de un autómata. El autómata debe ser finito y
determinista, abreviado AFD. Tras este proceso, se obtendrá una nueva versión
del autómata inicial, el autómata conexo equivalente.  


\subsection{Estructura}
Para representar un AFD, $M = \{ Q, \Sigma, s, F, \Delta \} $, se ha empleado 
una lista para cada uno de sus componentes, exceptuando al estado inicial $s$, 
que consta de un sólo elemento. Tanto un estado como un símbolo, están 
contituídos por una cadena de caracteres. Una transición contiene dos estados, 
inicial y final, y un símbolo.

\begin{lstlisting}
Objeto Automata:
	estados:Lista<String>
	alfabeto:Lista<String>
	estadosFinales:Lista<String>
	estadoInicial:String
	transiciones:Lista<Transicion>

Objeto Transicion:
	estadoInicial:String
	estadoFinal:String
	simbolo:String
\end{lstlisting}


\subsection{Cargar configuración}
Para cargar el autómata del fichero de configuración, se emplea el siguiente 
algoritmo.
\begin{lstlisting}
funcion inicializarAutomata(fichero configuracion):
	String linea = Primera linea(fichero configuracion)
	String lineas[] = Dividir linea por ";"
	estados = lineas[1];
	alfabeto = lineas[2];
	estadosFinales = lineas[3];
	estadoInicial = lineas[4];
	desde i = 5 hasta fin(lineas) hacer:
		transiciones += nueva transicion lineas[i];
\end{lstlisting}


\subsection{Algoritmo}
El algoritmo que calcula el autómata conexo equivalente, se muestra a 
continuación en pseudocódigo.
\begin{lstlisting}

funcion conexo(afd:Automata):
	listaEstados = Lista vacia
	listaEstados.add(afd.estadoInicial)
	nuevasTransiciones = Lista vacia
	visitados = Lista vacia
	while listaEstados != vacia:
		elem = listaEstados[0]
		visitados.add(elem)
		para trans en afd.transiciones:
			si (trans.ini == elem):
				nuevasTransiciones.add(trans)
				si ((trans.fin no esta en visitados) y
				(trans.fin no esta en listaEstados):
					listaEstados.add(trans.fin)
	afd.transiciones = nuevasTransiciones

\end{lstlisting}

%%%%%%%%%%%%%%%%%%%%%%%%%%%%%%%%%%%%%%%%%%%%%%%%%%%%%%%%%%%%%%%%%%%%%%%%%%%%%%%

\section{Lenguaje original (Python)}

\subsection{Versión y Sistema Operativo}
Python 2.7.6 y Ubuntu 14.04 de 64 bits.


\subsection{Características generales del lenguaje}

\begin{itemize}
\item Lenguaje de programación multiparadigma.
\item Soporta orientación a objetos.
\item Programación imperativa.
\item Programación funcional (en menor medida).
\item Lenguaje interpretado.
\item Tipado dinámico.
\item Multiplataforma.
\item Conteo de referencias (para eliminar bloques no usados).
\item Resolución dinámica de nombres.
\item Las variables se definen de forma dinámica.
\item Listas mutables (se puede cambiar su contenido en tiempo de ejecución).
\item Tuplas inmutables.
\item Listas y tuplas pueden contener elementos de diferentes tipos.
\end{itemize}


\subsection{Análisis detallado}

\subsubsection{Ventajas}
\begin{itemize}
\item Fácil lectura (casi pseudocódigo).
\item Listas iterables sin necesidad de iteradores.
\item Depuración sencilla en consola de modo interactivo.
\item Las listas pueden contener varios tipos de datos, hasta pueden contener 
tuplas.
\item Muy alto nivel.
\item La entrada salida es sencilla.
\end{itemize}

\subsubsection{Desventajas}
\begin{itemize}
\item No permite la gestión manual de la memoria.
\end{itemize}
  

\subsection{Justificaciones respecto a las decisiones de diseño e 
implementación}
Se ha diseñado el automata en un objeto, ya que permite mantener una
representación cercana a la que se plantea en pseudocódigo. Una transición es 
otro objeto, de forma que se almacena una lista de transiciones.


\subsection{Características que no se pudieron aprovechar del lenguaje}
No se ha usado la programación funcional, ni "list comprehension". Tampoco se 
empleó recursión en el algoritmo, aunque sí sería posible, realizando 
modificaciones. También se podrían haber utilizado conjuntos pero no todos los 
demás los soportan, por lo que se optó una implementación con listas.


%%%%%%%%%%%%%%%%%%%%%%%%%%%%%%%%%%%%%%%%%%%%%%%%%%%%%%%%%%%%%%%%%%%%%%%%%%%%%%%


\section{Lenguaje C}

\subsection{Compilador versión y el sistema operativo empleado}
Compilador:
\begin{lstlisting}
gcc (GCC) 4.9.1 20140903 (prerelease)
\end{lstlisting}
Sistema operativo:
\begin{lstlisting}
Linux 3.16.4-1 x86_64 GNU/Linux
\end{lstlisting}


\subsection{Características generales del lenguaje}
Es un lenguaje imperativo procedural y funcional. Se considera de medio nivel,
aunque incluye operaciones de bajo nivel, directamente escritas en ensamblador.

Emplea un tipado fuertemente estático que impide realizar operaciones sin
sentido. Sin embargo no tiene programación orientada a objetos, ni recolección
de basura.


\subsection{Análisis detallado}

\subsubsection{Ventajas}
\begin{itemize}
\item Incluye un preprocesador muy útil para crear macros.
\item Emplea punteros para la gestión de la memoria.
\item Contiene estructuras.
\item Permite la recursión.
\item Pocas palabras clave
\end{itemize}

\subsubsection{Desventajas}
\begin{itemize}
\item No tiene recolector de basura.
\item No implementa programación orientada a objetos.
\item No dipone de programación multihilo de forma nativa.
\item No tiene protección de memoria para accesos inválidos.
\end{itemize}


\subsection{Justificaciones respecto a las decisiones de diseño e 
implementación}
Para la implementación en C, se ha seguido la aproximación inicial, empleando
listas doblemente enlazadas. El autómata es una estructura, que contiene las
listas.

La implementación de las listas se ha logrado empleando punteros, y funciones
que ofrecen una interfaz para el acceso a los elementos.

Esta elección mantiene una gran similaritud con la primera aproximación.
Almacenar el autómata en forma de estructura, mantiene los elementos ordenados y
clasificados, facilitando su uso y comprensión.

La entrada y salida de la configuración del autómata se ha realizado usando la
entrada y salida estándar. Para ello se usan funciones como fgets y printf.


\subsection{Características que no se pudieron aprovechar del lenguaje}
La recursión, ya que la implementación original no la requería. Se podría haber
empleado, modificando el algoritmo, para incluirla, pero requeriría mantener una
lista de elementos ya visitados, para evitar lazos. Por lo demás el algoritmo
sería muy similar.

%%%%%%%%%%%%%%%%%%%%%%%%%%%%%%%%%%%%%%%%%%%%%%%%%%%%%%%%%%%%%%%%%%%%%%%%%%%%%%%

\section{Lenguaje ensamblador (NASM)}

\subsection{Compilador versión y el sistema operativo empleado}
Compilador:
\begin{lstlisting}
NASM version 2.11.05 compiled on May 31 2014
gcc (GCC) 4.9.1 20140903 (prerelease)
\end{lstlisting}
Sistema operativo:
\begin{lstlisting}
Linux 3.16.4-1 x86_64 GNU/Linux
\end{lstlisting}


\subsection{Características generales del lenguaje}
Ensamblador es una abstracción de las instrucciones de procesador, para que
resulten más memorizables. Además el compilador elegido, NASM, contiene
etiquetas, que evitan manejar directamente las posiciones de memoria.


\subsection{Análisis detallado}

\subsubsection{Ventajas}
\begin{itemize}
\item Permite construir código optimizado para una máquina específica.
\item Aprendizaje de la estructura interna del procesador.
\item Aprovechar la memoria caché y los registros.
\end{itemize}

\subsubsection{Desventajas}
\begin{itemize}
\item No tiene estructuras de control, sólo saltos.
\item Gestión de la memoria y pila de forma manual.
\item No implementa programación orientada a objetos.
\item No dipone de programación multihilo de forma nativa.
\item No tiene protección de memoria para accesos inválidos.
\item Transformar un problema de alto nivel a bajo nivel es muy complejo.
\item Programación muy lenta.
\item La depuración es un proceso costoso, y también muy lento.
\item Depende de la arquitectura de la máquina.
\end{itemize}


\subsection{Justificaciones respecto a las decisiones de diseño e 
implementación}
Es muy diferente al lenguaje inicial. Para su implementación de ha elegido sólo 
la parte del algoritmo que genera el autómata mínimo conexo. Las funciones de 
entrada de la configuración, así como las estructuras de datos, y listas, se han 
empleado del lenguaje C. Esta elección, es debida a la enorme complejidad en la 
que resultaría, realizar todo el programa en ensamblador.

El algoritmo es una traducción del programa en C a ensamblador. Cada línea es 
codificada con algunas instrucciones. Los bucles se forman con saltos y 
etiquetas.

%%%%%%%%%%%%%%%%%%%%%%%%%%%%%%%%%%%%%%%%%%%%%%%%%%%%%%%%%%%%%%%%%%%%%%%%%%%%%%%
  
\section{Java}

\subsection{Compilador versión y el sistema operativo empleado}
JAVA 8, Ubuntu 14.04 de 64 bits.


\subsection{Características generales del lenguaje}
\begin{itemize}
\item Lenguaje de programación de propósito general
\item Concurrente.
\item Orientado a objetos.
\item Recolector de basura (para evitar fugas de memoria).
\item Variables: no se definen de forma dinámica (se especifica el tipo antes de 
usarla).
\item Diseñado para ofrecer seguridad y portabilidad, y no ofrece acceso directo 
al hardware de la arquitectura ni al espacio de direcciones.
\item Recolector de basura puede suponer una carga para el sistema (sistemas en 
tiempo real).
\item No dispone de operadores de sobrecarga definidos por el usuario.
\item No es un lenguaje absolutamente orientado a objetos, no todos los valores 
son objetos.
\end{itemize}


\subsection{Análisis detallado}
Muy parecido al lenguaje original, con la diferencia de las listas, que 
necesitan un iterador para recorrerlas y no se pueden iterar directamente como 
en Python. Además la entrada y salida se debe manejar con excepciones y en 
general es un poco más laborioso que en Python. En cuanto a la orientacion a 
objetos es igual. Sólo se han cambiado las listas de Python por los arrayList de 
Java.


\subsection{Justificaciones respecto a las decisiones de diseño e 
implementación}
El autómata mantiene la misma estructura que en Python, modificando las listas 
por arrayList. El código Java puede ser a veces redundante en comparación con 
otros lenguajes.

Podrían haberse empleado conjuntos, igual que en Python, pero se ha decidido 
prescindir de ellos, en beneficio de otras implementaciones más sencillas con 
listas.


\subsection{Características que no se pudieron aprovechar del lenguaje}
Casi ninguna de las multiples librerías ya que no eran necesarias.

\end{document}
